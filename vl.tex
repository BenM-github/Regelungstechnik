
\documentclass{article}

\begin{document}

  \section*{15.10.2024}

  x) Regelgröße:
  - die physikalische Größe, die geregelt werden soll. Das bedeutet ein physikalischer Wert in einem gewünschten Maß gehalten wird.

  w) Führungsgröße:
  -

  y) Stellgröße:
  - physikalische Größe, welche die Regelgröße auf eine gewünschte Weise beeinflusst. (Bsp. Volumen Strom)

  e) Regelabweichung:
  - Differenz = Führungsgröße - Regelgröße

  z) Störgröße:
  - Einflüsse die selbst nicht beeinflusst werden können
  - Größen, die eine eingestellte Regelung aus dem Gleichgewicht bringt.

  Regelstrecke:
  - ist das zugrunde liegende System

  Systemarten: (Eingang/Ursache - Ausgang/Wirkung)
  - Intigrator: bsp. Volumenstrom wird in Volumen aufintigriert
  - Verstärker: bsp. Hebel
\end{document}
